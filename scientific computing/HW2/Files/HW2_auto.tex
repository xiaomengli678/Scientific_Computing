\documentclass[titlepage]{article}

\usepackage{amssymb,amsmath,amsthm}
\usepackage{graphicx} % Package for including figures
%\usepackage{psfrag,color}
\usepackage[utf8]{inputenc}
\usepackage{longtable}

\title{HW2 Data Report for Math/CS 471}
\author{Xiaomeng Li}
\date{09/16/2017}   % Activate to display a given date or no date


\begin{document}
\maketitle

\begin{abstract}
This is the HW2 report. This report is made all through LaTeX.
\end{abstract}

\section{Question 1}
We run the script and see the output. The iteration is ten times.

\section{Question 2}
Please see the newtonS.f90.Template file. The do-while loop is used
for approximating the absolute error.

\section{Question 3}
In newtonS.f90.Template file, I add linear and quadratic convergence
right after the x output. So for every iteration there should be a
total of 6 outputs.

\section{Question 4}
\subsection{Convergence}
The first number is linear convergence and the second is quadratic
convergence.  
Rate of convergence for 'x' is 0.5000000000000000E+00 and
0.5000000000000000E+00.
Rate of convergence for 'x*x' is 0.5000000000000000E+00 and 
0.1407374883553280E+15. 
Rate of convergence for 'sin(x)+cos(x*x)' is 0.1633240749027686E-04 and
0.7677507053309917E+00.
The results are read from the file when the iteration reaches maximum.

\subsection{Convergence discussion}
$f'(x)$ is ('1.d0', '2.d0*x', 'cos(x)-2.d0*x*sin(x*x)'). 
For the first one, when x comes closer to its root, there should not
be very big change happening since $f'(x)$ is already a
constant. However, when it comes to the second one, it is difficult to
determine convergence since x still exists in the equation. The third
is better since they are cos() and sin(), which is more possible to give a
final rate of convergence I think when it comes to quadratic
convergence. The data also proves this quadratic convergence when it
reaches fifth and sixth iteration in case three.

\subsection{Modified Newton’s method discussion}
I think I will implement Modified Newton’s method on case two since
the multiplicity m of the root is known in advance (I know what $f'(x)$
is as soon as I have a x). $-$$m\times($f'(x)$/$f(x)$)$ = $x_{n+1}$
$-$ $x_{n}$. So I will just modify newtonS.f90.Template file.
\section{Question 5}
$>>$ is actally more useful.
\section{Question 6}
One good method writing double slashes at the end of each output data row has
not come up though.
\section {Tabulated Data}
Explanation: Table 1: the first character is the case ( 'x', 'x*x' ,
'sin(x)+cos(x*x)'). The second number is the iteration number. The
third one is x value, the forth one is dx. Table 2: Explanation: the first
character is the case ( 'x', 'x*x' , 'sin(x)+cos(x*x)'). The second
numbe is the iteration number. The third one is linear convergence 
and the forth one is quadratic convergence.
 
\subsection{Table1without convergence}

\begin{centering}
\begin{longtable}[h]
 
\hline
 \hline
 x & 01 & 0.0000000000000000E+00 & 0.5000000000000000E+00 \\
 x & 02 & 0.0000000000000000E+00 & -0.0000000000000000E+00 \\
 x*x & 01 & -0.2500000000000000E+00 & 0.2500000000000000E+00 \\
 x*x & 02 & -0.1250000000000000E+00 & 0.1250000000000000E+00 \\
 x*x & 03 & -0.6250000000000000E-01 & 0.6250000000000000E-01 \\
 x*x & 04 & -0.3125000000000000E-01 & 0.3125000000000000E-01 \\
 x*x & 05 & -0.1562500000000000E-01 & 0.1562500000000000E-01 \\
 x*x & 06 & -0.7812500000000000E-02 & 0.7812500000000000E-02 \\
 x*x & 07 & -0.3906250000000000E-02 & 0.3906250000000000E-02 \\
 x*x & 08 & -0.1953125000000000E-02 & 0.1953125000000000E-02 \\
 x*x & 09 & -0.9765625000000000E-03 & 0.9765625000000000E-03 \\
 x*x & 10 & -0.4882812500000000E-03 & 0.4882812500000000E-03 \\
 x*x & 11 & -0.2441406250000000E-03 & 0.2441406250000000E-03 \\
 x*x & 12 & -0.1220703125000000E-03 & 0.1220703125000000E-03 \\
 x*x & 13 & -0.6103515625000000E-04 & 0.6103515625000000E-04 \\
 x*x & 14 & -0.3051757812500000E-04 & 0.3051757812500000E-04 \\
 x*x & 15 & -0.1525878906250000E-04 & 0.1525878906250000E-04 \\
 x*x & 16 & -0.7629394531250000E-05 & 0.7629394531250000E-05 \\
 x*x & 17 & -0.3814697265625000E-05 & 0.3814697265625000E-05 \\
 x*x & 18 & -0.1907348632812500E-05 & 0.1907348632812500E-05 \\
 x*x & 19 & -0.9536743164062500E-06 & 0.9536743164062500E-06 \\
 x*x & 20 & -0.4768371582031250E-06 & 0.4768371582031250E-06 \\
 x*x & 21 & -0.2384185791015625E-06 & 0.2384185791015625E-06 \\
 x*x & 22 & -0.1192092895507812E-06 & 0.1192092895507812E-06 \\
 x*x & 23 & -0.5960464477539062E-07 & 0.5960464477539062E-07 \\
 x*x & 24 & -0.2980232238769531E-07 & 0.2980232238769531E-07 \\
 x*x & 25 & -0.1490116119384766E-07 & 0.1490116119384766E-07 \\
 x*x & 26 & -0.7450580596923828E-08 & 0.7450580596923828E-08 \\
 x*x & 27 & -0.3725290298461914E-08 & 0.3725290298461914E-08 \\
 x*x & 28 & -0.1862645149230957E-08 & 0.1862645149230957E-08 \\
 x*x & 29 & -0.9313225746154785E-09 & 0.9313225746154785E-09 \\
 x*x & 30 & -0.4656612873077393E-09 & 0.4656612873077393E-09 \\
 x*x & 31 & -0.2328306436538696E-09 & 0.2328306436538696E-09 \\
 x*x & 32 & -0.1164153218269348E-09 & 0.1164153218269348E-09 \\
 x*x & 33 & -0.5820766091346741E-10 & 0.5820766091346741E-10 \\
 x*x & 34 & -0.2910383045673370E-10 & 0.2910383045673370E-10 \\
 x*x & 35 & -0.1455191522836685E-10 & 0.1455191522836685E-10 \\
 x*x & 36 & -0.7275957614183426E-11 & 0.7275957614183426E-11 \\
 x*x & 37 & -0.3637978807091713E-11 & 0.3637978807091713E-11 \\
 x*x & 38 & -0.1818989403545856E-11 & 0.1818989403545856E-11 \\
 x*x & 39 & -0.9094947017729282E-12 & 0.9094947017729282E-12 \\
 x*x & 40 & -0.4547473508864641E-12 & 0.4547473508864641E-12 \\
 x*x & 41 & -0.2273736754432321E-12 & 0.2273736754432321E-12 \\
 x*x & 42 & -0.1136868377216160E-12 & 0.1136868377216160E-12 \\
 x*x & 43 & -0.5684341886080801E-13 & 0.5684341886080801E-13 \\
 x*x & 44 & -0.2842170943040401E-13 & 0.2842170943040401E-13 \\
 x*x & 45 & -0.1421085471520200E-13 & 0.1421085471520200E-13 \\
 x*x & 46 & -0.7105427357601002E-14 & 0.7105427357601002E-14 \\
 x*x & 47 & -0.3552713678800501E-14 & 0.3552713678800501E-14 \\
 x*x & 48 & -0.1776356839400250E-14 & 0.1776356839400250E-14 \\
 x*x & 49 & -0.8881784197001252E-15 & 0.8881784197001252E-15 \\
 sin(x)+cos(x*x) & 01 & -0.9351046647281536E+00 & -0.4351046647281536E+00 \\
 sin(x)+cos(x*x) & 02 & -0.8546415960180649E+00 & 0.8046306871008869E-01 \\
 sin(x)+cos(x*x) & 03 & -0.8493901358009870E+00 & 0.5251460217077924E-02 \\
 sin(x)+cos(x*x) & 04 & -0.8493688627401134E+00 & 0.2127306087358230E-04 \\
 sin(x)+cos(x*x) & 05 & -0.8493688623926731E+00 & 0.3474402480610000E-09 \\
 sin(x)+cos(x*x) & 06 & -0.8493688623926731E+00 & -0.0000000000000000E+00 \\
  
 \hline
\hline

\end{longtable}
\end{centering}

\subsection{Table2with convergence}

\begin{centering}
\begin{longtable}[h]
 
\hline
 \hline
 x & 01 & 0.1000000000000000E+01 & 0.1000000000000000E+01 \\
 x & 02 & 0.5000000000000000E+00 & 0.5000000000000000E+00 \\
 x*x & 01 & 0.1000000000000000E+01 & 0.1000000000000000E+01 \\
 x*x & 02 & 0.2500000000000000E+00 & 0.2500000000000000E+00 \\
 x*x & 03 & 0.5000000000000000E+00 & 0.2000000000000000E+01 \\
 x*x & 04 & 0.5000000000000000E+00 & 0.4000000000000000E+01 \\
 x*x & 05 & 0.5000000000000000E+00 & 0.8000000000000000E+01 \\
 x*x & 06 & 0.5000000000000000E+00 & 0.1600000000000000E+02 \\
 x*x & 07 & 0.5000000000000000E+00 & 0.3200000000000000E+02 \\
 x*x & 08 & 0.5000000000000000E+00 & 0.6400000000000000E+02 \\
 x*x & 09 & 0.5000000000000000E+00 & 0.1280000000000000E+03 \\
 x*x & 10 & 0.5000000000000000E+00 & 0.2560000000000000E+03 \\
 x*x & 11 & 0.5000000000000000E+00 & 0.5120000000000000E+03 \\
 x*x & 12 & 0.5000000000000000E+00 & 0.1024000000000000E+04 \\
 x*x & 13 & 0.5000000000000000E+00 & 0.2048000000000000E+04 \\
 x*x & 14 & 0.5000000000000000E+00 & 0.4096000000000000E+04 \\
 x*x & 15 & 0.5000000000000000E+00 & 0.8192000000000000E+04 \\
 x*x & 16 & 0.5000000000000000E+00 & 0.1638400000000000E+05 \\
 x*x & 17 & 0.5000000000000000E+00 & 0.3276800000000000E+05 \\
 x*x & 18 & 0.5000000000000000E+00 & 0.6553600000000000E+05 \\
 x*x & 19 & 0.5000000000000000E+00 & 0.1310720000000000E+06 \\
 x*x & 20 & 0.5000000000000000E+00 & 0.2621440000000000E+06 \\
 x*x & 21 & 0.5000000000000000E+00 & 0.5242880000000000E+06 \\
 x*x & 22 & 0.5000000000000000E+00 & 0.1048576000000000E+07 \\
 x*x & 23 & 0.5000000000000000E+00 & 0.2097152000000000E+07 \\
 x*x & 24 & 0.5000000000000000E+00 & 0.4194304000000000E+07 \\
 x*x & 25 & 0.5000000000000000E+00 & 0.8388608000000000E+07 \\
 x*x & 26 & 0.5000000000000000E+00 & 0.1677721600000000E+08 \\
 x*x & 27 & 0.5000000000000000E+00 & 0.3355443200000000E+08 \\
 x*x & 28 & 0.5000000000000000E+00 & 0.6710886400000000E+08 \\
 x*x & 29 & 0.5000000000000000E+00 & 0.1342177280000000E+09 \\
 x*x & 30 & 0.5000000000000000E+00 & 0.2684354560000000E+09 \\
 x*x & 31 & 0.5000000000000000E+00 & 0.5368709120000000E+09 \\
 x*x & 32 & 0.5000000000000000E+00 & 0.1073741824000000E+10 \\
 x*x & 33 & 0.5000000000000000E+00 & 0.2147483648000000E+10 \\
 x*x & 34 & 0.5000000000000000E+00 & 0.4294967296000000E+10 \\
 x*x & 35 & 0.5000000000000000E+00 & 0.8589934592000000E+10 \\
 x*x & 36 & 0.5000000000000000E+00 & 0.1717986918400000E+11 \\
 x*x & 37 & 0.5000000000000000E+00 & 0.3435973836800000E+11 \\
 x*x & 38 & 0.5000000000000000E+00 & 0.6871947673600000E+11 \\
 x*x & 39 & 0.5000000000000000E+00 & 0.1374389534720000E+12 \\
 x*x & 40 & 0.5000000000000000E+00 & 0.2748779069440000E+12 \\
 x*x & 41 & 0.5000000000000000E+00 & 0.5497558138880000E+12 \\
 x*x & 42 & 0.5000000000000000E+00 & 0.1099511627776000E+13 \\
 x*x & 43 & 0.5000000000000000E+00 & 0.2199023255552000E+13 \\
 x*x & 44 & 0.5000000000000000E+00 & 0.4398046511104000E+13 \\
 x*x & 45 & 0.5000000000000000E+00 & 0.8796093022208000E+13 \\
 x*x & 46 & 0.5000000000000000E+00 & 0.1759218604441600E+14 \\
 x*x & 47 & 0.5000000000000000E+00 & 0.3518437208883200E+14 \\
 x*x & 48 & 0.5000000000000000E+00 & 0.7036874417766400E+14 \\
 x*x & 49 & 0.5000000000000000E+00 & 0.1407374883553280E+15 \\
 sin(x)+cos(x*x) & 01 & 0.1000000000000000E+01 & 0.1000000000000000E+01 \\
 sin(x)+cos(x*x) & 02 & 0.4351046647281536E+00 & 0.4351046647281536E+00 \\
 sin(x)+cos(x*x) & 03 & 0.1849280764672121E+00 & 0.4250197514723320E+00 \\
 sin(x)+cos(x*x) & 04 & 0.6526547273506428E-01 & 0.8111233362254442E+00 \\
 sin(x)+cos(x*x) & 05 & 0.4050884895680481E-02 & 0.7713825732711236E+00 \\
 sin(x)+cos(x*x) & 06 & 0.1633240749027686E-04 & 0.7677507053309917E+00 \\
  
 \hline
\hline

\end{longtable}
\end{centering}
\end{document}


